% Template for Elsevier submission with R Markdown

% Stuff changed from PLOS Template
\documentclass[authoryear, review]{elsarticle}
\usepackage[american]{babel}
\usepackage[section]{placeins}

\bibliographystyle{model5-names}

\journal{Cognition}



% amsmath package, useful for mathematical formulas
\usepackage{amsmath}
% amssymb package, useful for mathematical symbols
\usepackage{amssymb}

% hyperref package, useful for hyperlinks
\usepackage{hyperref}

% graphicx package, useful for including eps and pdf graphics
% include graphics with the command \includegraphics
\usepackage{graphicx}

% Sweave(-like)
\usepackage{fancyvrb}
\DefineVerbatimEnvironment{Sinput}{Verbatim}{fontshape=sl}
\DefineVerbatimEnvironment{Soutput}{Verbatim}{}
\DefineVerbatimEnvironment{Scode}{Verbatim}{fontshape=sl}
\newenvironment{Schunk}{}{}
\DefineVerbatimEnvironment{Code}{Verbatim}{}
\DefineVerbatimEnvironment{CodeInput}{Verbatim}{fontshape=sl}
\DefineVerbatimEnvironment{CodeOutput}{Verbatim}{}
\newenvironment{CodeChunk}{}{}

% cite package, to clean up citations in the main text. Do not remove.
\usepackage{cite}

\usepackage{color}

% Use doublespacing - comment out for single spacing
%\usepackage{setspace}
%\doublespacing


% % Text layout
% \topmargin 0.0cm
% \oddsidemargin 0.5cm
% \evensidemargin 0.5cm
% \textwidth 16cm
% \textheight 21cm

% Bold the 'Figure #' in the caption and separate it with a period
% Captions will be left justified
\usepackage[labelfont=bf,labelsep=period,justification=raggedright]{caption}


% Remove brackets from numbering in List of References
\makeatletter
\renewcommand{\@biblabel}[1]{\quad#1.}
\makeatother


% Leave date blank
\date{}

\begin{document}

\begin{frontmatter}

\title{Developmental Changes in the Relationship Between Grammar and the
Lexicon}

\author[mb]{\corref{cor}Mika Braginsky}
\cortext[cor]{Corresponding author}
\ead{mikabr@stanford.edu}
\author[dy]{Daniel Yurovsky}
\author[vam]{Virginia A. Marchman}
\author[mcf]{Michael C. Frank}
\address{Department of Psychology, Stanford University, United States}


\begin{abstract}
How does abstract structure emerge during language learning? On some
accounts, children's early syntax emerges from direct generalizations
from particular lexical items, while on others, syntactic structure is
acquired independently and follows its own timetable. Progress on
differentiating these views requires detailed developmental data. Using
parental reports of vocabulary and grammar abilities, previous analyses
have shown that early syntactic abstraction strongly depends on the
growth of the lexicon, providing support for lexicalist and emergentist
theories. Leveraging a large cross-linguistic dataset, we replicate and
extend these findings, demonstrating similar patterns in each of four
languages. Moreover, the power of our dataset reveals that there are
measurable effects of age over and above those attributable to
vocabulary size, and that these effects are greater for aspects of
language ability more closely tied to syntax than morphology. These
findings suggest non-lexical contributions to the growth of syntactic
abstraction that all theories must address.
\end{abstract}

\begin{keyword}

\end{keyword}

\end{frontmatter}

\section{Introduction}\label{introduction}

A child as young as two or three (who happens to be acquiring English)
can hear someone say \emph{Alice glipped the blicket} and draw a wealth
of inferences from the morphological and syntactic structure of that
utterance: that \emph{Alice} and \emph{blicket} are entities in the
world and \emph{glipping} is an action; that Alice is the one glipping
and the blicket is the thing being glipped; that glipping occurred in
the past (rather than the present, as in \emph{Alice is glipping the
blicket}); that a single blicket was involved (rather than multiple, as
in \emph{Alice glipped the blickets}). What mechanisms underlie the
formation of generalizations that support such inferences? Does an
understanding of the abstract structure of language emerge from the
interactions of individual words, or is structure acquired and
represented separately?

On nativist theories like principles and parameters (Chomsky
\hyperref[ref-chomsky1981]{1981}, Baker
(\hyperref[ref-baker2005]{2005})), grammar emerges independently from
lexical knowledge following its own, largely maturational, timetable.
According to lexicalist theories, in contrast, grammatical structure
emerges from graded generalizations on the basis of lexical items, and
at least early in development, there may be little or no representation
of morphological and syntactic rules or regularities \emph{per se}
(Tomasello \hyperref[ref-tomasello2003]{2003}). Even when syntactic
structures are eventually represented, these representations are
directly related to more concrete lexical structure (Bannard, Lieven,
and Tomasello \hyperref[ref-bannard2009]{2009}). Therefore, grammatical
development should be tightly yoked to lexical development (E. Bates and
Goodman \hyperref[ref-bates1999]{1999}). Data on the relationship
between the lexicon, grammar, and age are important for informing this
fundamental theoretical debate.

One source of such data is the MacArthur-Bates Communicative Development
Inventory (CDI), a widely-used assessment tool in which parents report
which words their child produces on a checklist organized by
lexical-semantic categories. Children's vocabulary size can thus be
estimated over the entire checklist, or for sub-categories. The CDI also
provides indices of grammar learning by asking about children's use of
inflected forms (e.g., \emph{walked}) and the complexity of their word
combinations (e.g., \emph{kitty sleeping / kitty is sleeping}).
Influential early findings using this measure showed that early
vocabularies tend to be composed primarily of nouns, while verbs and
closed-class forms, which might support the transition into complex
sentences, are typically acquired later (Elizabeth Bates et al.
\hyperref[ref-bates1994]{1994}). Further, across different populations
and languages, global estimates of grammatical development are more
strongly predicted by overall vocabulary size than by age, providing
support for lexicalist theories (see (E. Bates and Goodman
\hyperref[ref-bates1999]{1999}) for a review).

While impressive in their time, the scope and power of these early
studies were limited, relying on relatively small norming samples
(1000--2000 children) with few opportunities for direct comparisons of
the nature or extent of these relations across languages. The current
study addresses these limitations by using data from Wordbank
(wordbank.stanford.edu), a new web-based tool that aggregates
pre-existing samples of CDI data into a consistent format across forms
and languages. While still in development, the resulting database is
already considerably larger than those previously available, and thus
allows analyses of lexical-grammar-age relations with enhanced
statistical power and broader cross-linguistic representation.

In the current study, we present data from 19,822 children aged 16--32
months, using adaptations of the CDI Words \& Sentences form in four
languages: English, Spanish, Norwegian, and Danish. We replicate classic
findings of strong lexicon-grammar relations and patterns of vocabulary
composition across four languages. We also extend these findings through
novel analyses afforded by the Wordbank database.

We explore a hypothesis that was not explicitly tested in these earlier
studies: that there remains age-related variance in grammatical
development unexplained by vocabulary development. While the overall
relationship between grammar and the lexicon provides support for
lexicalist theories, the identification of age-related variance would
suggest the presence of developmental processes that regulate grammar
learning, above and beyond those captured by measures of vocabulary
size. Such age-related processes could be either maturational or
experiential, and either domain-general (like working memory) or
language-specific (like grammatical competence). Since both nativist and
constructivist theories could in principle predict age-linked variance
in grammatical development, our goal in the current work is not to
differentiate between these theories, but instead to test this novel
prediction and explore its implications for future work on understanding
the processes of grammatical development.

An additional contribution of our work is that due to the size of our
dataset, we are able to make more fine-grained distinctions than the
initial cut between grammar and the lexicon. In particular, we
distinguish morphology from multi-word syntax, since morphological
generalizations might be more specifically dependent on vocabulary size
than those requiring more global, sentence-level syntactic regularities.
Similarly, we distinguish age-related contributions to different parts
of the vocabulary. Lexical items like verbs often require some syntactic
information to learn (Gleitman \hyperref[ref-gleitman1990]{1990}) and
hence might be more linked to age-related factors that extend beyond
vocabulary size.

The outline of the paper is as follows. We begin by describing the
Wordbank database, the CDI measures, and our general analytic approach.
We then describe two sets of analyses exploring the contribution of age
to lexicon-grammar links (Analysis 1) and to patterns of vocabulary
composition (Analysis 2). In Analysis 1, we delineate the grammar
sections into items that reflect a broad distinction between
inflectional morphology vs.~sentence-level syntactic knowledge. We
expect that age-related contributions to grammar should be evident to a
larger extent for syntax than morphology. In Analysis 2, we further
leverage this technique to determine if age-related contributions vary
across word classes. In particular, we predict that acquisition of
predicates (verbs and adjectives) and function words should be
relatively more dependent on syntactic factors than acquisition of
nouns, and thus should exhibit a greater relative influence of age.
These analyses reveal greater effects of age on aspects of grammar that
are more aligned with syntax than with morphology, and greater effects
of age on function words and perhaps on predicates than on nouns. In the
General Discussion, we consider potential domain-specific and
domain-general explanations consistent with these findings.

\section{Methods}\label{methods}

\subsection{CDI Form Database}\label{cdi-form-database}

We developed Wordbank, a structured database of CDI data, to aggregate
and archive CDI data across languages and labs and facilitate easy
querying and analysis. By collecting language development data at an
unprecedented scale, Wordbank enables the exploration of novel
hypotheses about the course of lexical and grammatical development. At
the time of writing, Wordbank included data on four languages: English
(Fenson et al. \hyperref[ref-fenson2007]{2007}), Spanish
(Jackson-Maldonado et al. \hyperref[ref-jackson1993]{1993}), Norwegian
(Simonsen et al. \hyperref[ref-simonsen2014]{2014}), and Danish (Bleses
et al. \hyperref[ref-bleses2008]{2008}), with both cross-sectional and
longitudinal data. This dataset encompasses norming data from each
language as well as a number of smaller-scale studies, some of which did
not provide data from the grammar sections.

\subsection{CDI Measures}\label{cdi-measures}

In all four languages, the CDI forms contain both vocabulary checklists
and other questions relevant to the child's linguistic development. All
of the data reported here come from the Words \& Sentences form,
administered to children ages 16--32 months. Each of these instruments
includes a Vocabulary section, which asks whether the child produces
each of around 700 words from a variety of semantic and syntactic
categories (e.g., \emph{foot}, \emph{run}, \emph{so}); a Word Form
section, which asks whether the child produces each of around 30
morphologically inflected forms of nouns and verbs (e.g., \emph{feet},
\emph{ran}); and a Complexity section, which asks whether the child's
speech is most similar to the syntactically simpler or more complex
versions of around 40 sentences (e.g., \emph{two foot / two feet},
\emph{there a kitty / there's a kitty}). Each language's instrument is
not just a translation of the English form, but rather was constructed
and normed to reflect the lexicon and grammar of that language.

To analyze lexical and grammatical development, we derive several
measures. Each child's vocabulary size is computed as the proportion of
words on the corresponding CDI form that the child is reported to
produce. Similarly, each child's Word Form score is the proportion of
word forms they are reported to produce, and their Complexity score is
the proportion of complexity items for which they are reported to use
the more complex form. We compute all of these quantities as proportions
to make the scales comparable across languages.

\section{Analysis}\label{analysis}

By two years, most children have a sizable working vocabulary, including
verbs, prepositions, and closed class forms that perform grammatical
work. They are also beginning to use multi-word combinations (e.g.,
\emph{mommy sock}) and may demonstrate productive use of inflectional
morphemes (e.g., past tense \emph{-ed}). Previous studies have found a
strong connection between the size of the lexicon and grammatical
development as measured by the Complexity section, in many languages
including English, Italian, Hebrew, and Spanish (E. Bates and Goodman
\hyperref[ref-bates1999]{1999}). However, no study has had the power and
cross-linguistic representation to go beyond this initial finding to
explore relations to grammatical items that vary in
morphological/syntactic features. We extend this work by examining
grammatical development using two measures: the Word Form checklist as a
window into morphology and the Complexity checklist as a window into
syntax. For each measure, we investigate the effects of vocabulary size
and age.

\subsection{Results}\label{results}

We wanted to estimate how much variance in children's syntactic and
morphological development remains after accounting for that child's
vocabulary size. Specifically, we asked whether age provides additional
predictive power beyond vocabulary size. To estimate this effect, we fit
logistic regression models to each child's Word Form and Complexity
scores, predicting score as a function of vocabulary size and age in
months. For all languages and measures, the evidence is overwhelmingly
in favor of the model using both vocabulary and age as predictors, as
compared to the model using only vocabulary (the smallest difference in
AIC is 76).

Figure \emph{grammar\_predict\_plot} shows the data and models: each
point represents a child's score on a measure, while curves show the
relationship between score and vocabulary size. For all languages, the
curves for Word Form are nearly overlapping, showing little
differentiation across age groups. This indicates only small
contributions of age above and beyond vocabulary. In contrast, the
curves for Complexity show a characteristic fan across age groups,
indicating that the relationship between vocabulary size and complexity
score is modulated by age.

\begin{CodeChunk}
\begin{figure}
\includegraphics{figs/plot_grammar_predict-1} \caption[Each point shows an individual child, indicating their total vocabulary size and Word Form or Complexity score, with color showing their age bin (English n=4137]{Each point shows an individual child, indicating their total vocabulary size and Word Form or Complexity score, with color showing their age bin (English n=4137; Spanish n=1094; Norwegian n=8505; Danish n=2074). Panels show different languages, and curves are logistic regression models fit separately for each language and measure. The models were specified as score ~ vocab + age.}\label{fig:plot_grammar_predict}
\end{figure}
\end{CodeChunk}

Because of the size of our samples, all main effects and interactions
are highly significant. To assess the extent of the age contribution to
children's morphological and syntactic development, we compared the
coefficients of Word Form and Complexity models. Figure
\emph{grammar\_diffs\_plot} shows the coefficient of the age effect for
each measure across languages. In each language, the age effect
coefficient is substantially larger for Complexity than for Word Form,
indicating a greater age effect on those items that generally align with
syntax than morphology.

\begin{CodeChunk}

\includegraphics{figs/plot_grammar_diffs-1} \end{CodeChunk}

\subsubsection{By-Item}\label{by-item}

Given the heterogeneous nature of the CDI instruments, particularly in
the Complexity sections, we further broke down these items by
classifying them as capturing more morphological or more syntactic
phenomena. Items for which the difference between the simple and complex
sentences is in the inflection of a noun or verb (such as \emph{doggie
kiss me / doggie kissed me}) were coded as Morphological. The remainder
of the items were coded as Syntactic, since they involve the use of some
sentence-level syntactic construction (such as \emph{doggie table /
doggie on table}).

We then fit predictive models as above separately for every item. Figure
\emph{item\_area\_diffs} shows the age effect coefficient for each item.
In general, there is a three-way split: age effects are smallest for
Word Form items, then Morphological Complexity items, and largest for
Syntactic Complexity items, suggesting that more syntactic phenomena
have greater age contributions.

\begin{CodeChunk}

\includegraphics{figs/plot_item_diffs-1} \end{CodeChunk}

\subsection{Discussion}\label{discussion}

Building on previous analyses that showed a strong relationship between
lexical and grammatical development, we incorporated age into this
relationship. Across languages, our measures of syntactic development
consistently showed greater age modulation than measures of
morphological development. Further, distinguishing between items that
were more reflective of syntax than morphology, we again found greater
age effects for more syntactic items. Thus, this analysis provides
evidence for a relationship between syntactic development and age
\emph{not} captured by lexical development.

\section{General Discussion}\label{general-discussion}

The current study revisits classic findings but also explores novel
questions regarding lexicon-grammar relations and vocabulary composition
through Wordbank, a newly-developed web-based tool for cross-linguistic
analyses of large CDI datasets. Our results provided general support for
a lexicalist view, in that, in four languages, variance in vocabulary
production strongly aligned with variance in grammar. However, we also
estimated additional age-related contributions, specifically contrasting
the links to morphological forms vs.~syntactic constructions, and to
different lexical categories. In general, we find that measures of
grammar that are more closely aligned with syntax are modulated by age
to a greater extent than those reflecting morphology. Also, we find that
the trajectories of predicate and function word representation in the
vocabulary are modulated by age to a greater extent than noun
representation (albeit with some variability across languages). Both
findings suggest a place for developmental processes that facilitate
grammatical acquisition beyond pure lexical growth.

Our analyses suggest interesting new areas of research regarding
possible mechanisms driving children's early lexical development and how
those mechanisms might support children's transition from single words
to more grammatically complex utterances. One possibility is that these
developments are dependent on maturational factors that operate on
grammatical development in a domain-specific way, independent of
lexical-semantic processes. Another possibility is that age-related
effects represent more domain-general learning mechanisms, such as
attention or working memory, that provide differential support for
sentence-level processes than word-internal ones (Gathercole and
Baddeley \hyperref[ref-gathercole2014]{2014}). Future studies should
also explore the extent to which lexical and age-related processes are
shaped, either independently or in tandem, by features of the learning
environments that children experience (Weisleder and Fernald
\hyperref[ref-weisleder2013]{2013}).

Questions about the nature of grammatical representations in early
language have often seemed deadlocked. But by mapping out developmental
change across large samples and multiple languages, our findings here
challenge theories across the full range of perspectives to more fully
describe the mechanistic factors underlying the interaction of
vocabulary, grammar, and development.

\section{Acknowledgments}\label{acknowledgments}

Thanks to the MacArthur-Bates CDI Advisory Board, Dorthe Bleses,
Kristian Kristoffersen, Rune Nörgaard Jörgensen, and the members of the
Language and Cognition Lab.

\section*{References}\label{references}
\addcontentsline{toc}{section}{References}

\hyperdef{}{ref-baker2005}{\label{ref-baker2005}}
Baker, Mark C. 2005. ``Mapping the Terrain of Language Learning.''
\emph{Language Learning and Development} 1 (1). Taylor \& Francis:
93--129.

\hyperdef{}{ref-bannard2009}{\label{ref-bannard2009}}
Bannard, C., E. Lieven, and M. Tomasello. 2009. ``Modeling Children's
Early Grammatical Knowledge.'' \emph{Proceedings of the National Academy
of Sciences} 106 (41). National Acad Sciences: 17284.

\hyperdef{}{ref-bates1999}{\label{ref-bates1999}}
Bates, E., and J. Goodman. 1999. ``On the Emergence of Grammar from the
Lexicon.'' In \emph{The Emergence of Language}, edited by Brian
Macwhinney. Mahwah, NJ: Lawrence Erlbaum Associates.

\hyperdef{}{ref-bates1994}{\label{ref-bates1994}}
Bates, Elizabeth, Virginia Marchman, Donna Thal, Larry Fenson, Philip
Dale, J Steven Reznick, Judy Reilly, and Jeff Hartung. 1994.
``Developmental and Stylistic Variation in the Composition of Early
Vocabulary.'' \emph{Journal of Child Language} 21 (01). Cambridge Univ
Press: 85--123.

\hyperdef{}{ref-bleses2008}{\label{ref-bleses2008}}
Bleses, Dorthe, Werner Vach, Malene Slott, Sonja Wehberg, Pia Thomsen,
Thomas O Madsen, and Hans Basbøll. 2008. ``The Danish Communicative
Developmental Inventories: Validity and Main Developmental Trends.''
\emph{Journal of Child Language} 35 (03). Cambridge Univ Press: 651--69.

\hyperdef{}{ref-chomsky1981}{\label{ref-chomsky1981}}
Chomsky, N. 1981. ``Principles and Parameters in Syntactic Theory.''
\emph{Explanation in Linguistics: The Logical Problem of Language
Acquisition}. London: Longman, 1981b, 32--75.

\hyperdef{}{ref-fenson2007}{\label{ref-fenson2007}}
Fenson, Larry, Elizabeth Bates, Philip S Dale, Virginia A Marchman, J
Steven Reznick, and Donna J Thal. 2007. \emph{MacArthur-Bates
Communicative Development Inventories}.

\hyperdef{}{ref-gathercole2014}{\label{ref-gathercole2014}}
Gathercole, S. E., and A. D. Baddeley. 2014. \emph{Working Memory and
Language Processing}. Psychology Press.

\hyperdef{}{ref-gleitman1990}{\label{ref-gleitman1990}}
Gleitman, L. 1990. ``The Structural Sources of Verb Meanings.''
\emph{Language Acquisition}, 3--55.

\hyperdef{}{ref-jackson1993}{\label{ref-jackson1993}}
Jackson-Maldonado, Donna, Donna Thal, Virginia Marchman, Elizabeth
Bates, and Vera Gutiérrez-Clellen. 1993. ``Early Lexical Development in
Spanish-Speaking Infants and Toddlers.'' \emph{Journal of Child
Language} 20 (03). Cambridge Univ Press: 523--49.

\hyperdef{}{ref-simonsen2014}{\label{ref-simonsen2014}}
Simonsen, Hanne Gram, Kristian E Kristoffersen, Dorthe Bleses, Sonja
Wehberg, and Rune N Jørgensen. 2014. ``The Norwegian Communicative
Development Inventories: Reliability, Main Developmental Trends and
Gender Differences.'' \emph{First Language} 34 (1). SAGE Publications:
3--23.

\hyperdef{}{ref-tomasello2003}{\label{ref-tomasello2003}}
Tomasello, M. 2003. \emph{Constructing a Language: A Usage-Based Theory
of Language Acquisition}. Harvard University Press.

\hyperdef{}{ref-weisleder2013}{\label{ref-weisleder2013}}
Weisleder, Adriana, and Anne Fernald. 2013. ``Talking to Children
Matters: Early Language Experience Strengthens Processing and Builds
Vocabulary.'' \emph{Psychological Science} 24 (11). Sage Publications:
2143--52.

\bibliography{CogSci.bib}

\end{document}
